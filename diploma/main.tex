\documentclass{mipt-thesis-bs}
% Следующие две строки нужны только для biblatex. Для inline-библиографии их следует убрать.
\usepackage{mipt-thesis-biblatex}
\usepackage[justification=centering]{caption}
\usepackage{glossaries}
\usepackage{url}
\usepackage{pdfpages}
\addbibresource{diplom.bib}


\title{Применение генеративного моделирования как инструмента постановки естественнонаучной задачи педагогической направленности}
\author{Машалов Никита}
\supervisor{Щербаков Дмитрий}
%\referee{Петров Д.\,Е.}       % требуется только для mipt-thesis-ms
\groupnum{БО2-886}
\faculty{Физтех-школа физики и исследований им. Ландау}
\department{Кафедра инновационной педагогики}


\begin{document}

Автоматическая постановка и дополнение обучающих задач востребованное направление в сфере образования. 
Предлагается способы разрешения классических образования, включающий недобросовестную кооперации обучающихся во время тестирования, создания банка задач для инновационных образовательных курсов,
индвидуальном контроле знаний, Работы также показывают успешное применение автоматической генерации для формирования индивидуальной образовательной траектории

 тематически однородного, но разноуровневого по сложности. Появляются возможности генерации задач тематических задач по интересам обучающегося, обращихся к предметному опыту и интуиции.

объективного оцениванияне на  

Стремительное развитие генеративного моделирования 
в областях естественного языка \сite{radford2019language} и
визуальных изображений \cite{rombach2022highresolution}\cite{song2020generative} определили
новые подходы к задачам нотариального консультирования.

Задача работы применить методы генеративного моделирования для решения задачи. В дополнение работы 
выпускается кодовая база и обучающие данные для воспроизведения эксперимента и проведения
\footnone{https://huggingface.co/NMashalov} 


позволяет
На практике 


Недавние исследования показывают
успешное применение алгоритмов 

\chapter{Тематический обзор}



\chapter{Описание подхода}

В этом разделе будет проведено описание шагов, проделанных для описания


\chapter{Введение}

В рамках секции будут описаны методы, примененные для решения задачи генерации задач.


\begin{figure}[h]
    \centering
    \includegraphics[width=0.5\textwidth]{assets/Intefer_pic.png}
    \caption{Моделирование интерференционного изображения монохроматического источника}
    \label{fig_NewtonRings}
\end{figure}


\subsection{Обработка естественного языка}





\subsubsection{Методы обработки естественного языка}

Анализ естественного языка это межпредметная дисциплина.
Компьютерная лингвистика

Практически востребованной оказалась дистрибутивная гипотеза \сite{Schutze},
легшая в основу алгоритма \cite{NIPS2013_9aa42b31}


**Лемматизация** - процесс приведения языка к нормальной форме.

**


\subsubsection{Использование нейросетевых подходов}

В рамках раздела будет последовательно изложена хронология подходов
для построения генеративных моделей языка.

 модели строились на n-граммах \cite{heafield-2011-kenlm}

 


В последствии подходы развивились примением реккурентных нейронных сетей LSTM \сite{HochSchm97} и GRU



С эффективным примением архитектуры нейронной сети Attention \cite{NIPS2017_3f5ee243}, позволяющей эффективно обучать нейронные сети на графических ускорителях. 





\subsection{Построение обучающей задачи}







Наибольшей успех в обработке естественного языка связан с
введением 

Авторегрессионая модель

Подготовка датасета. 


\printbib

\end{document}



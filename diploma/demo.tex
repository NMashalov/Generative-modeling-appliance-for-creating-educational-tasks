\documentclass{mipt-thesis-bs}
% Следующие две строки нужны только для biblatex. Для inline-библиографии их следует убрать.
\usepackage{mipt-thesis-biblatex}
\usepackage[justification=centering]{caption}
\usepackage{glossaries}
\usepackage{url}
\usepackage{pdfpages}
\addbibresource{Diplom.bib}


\title{Обработка данных наземной калибровки фурье-спектрометра ФАСТ (ЭкзоМарс-2022)}
\author{Машалов Никита Евгеньевич}
\supervisor{ Шакун Алексей Владимирович,
	канд. физ.-мат. наук}
%\referee{Петров Д.\,Е.}       % требуется только для mipt-thesis-ms
\groupnum{БО2-886}
\faculty{Физтех-школа физики и исследований им. Ландау}
\department{Кафедра космической физики}


\documentclass[12pt,a4paper]{article}

\begin{document}

in free access on Github


Proposed work was built at the intersection of pedagogy and machine learning. 

Was provided with convenient UI interface 

\chapter{Introduction}

1. 

Some key viewpoints on datasets collection were introduced.

\begin{itemize}
    \item usage of high quality data;
    \item ;
\end{itemize};

2. Dataset collection


Following 

Despite being Hopefully were finacially accessible 

In further chapter will look through advances of 


3. Model 

Following options were considered


\begin{center}
\begin{tabular}{||c c c c||} 
     \hline
     Col1 & Col2 & Col2 & Col3 \\ [0.5ex] 
     \hline\hline
     1 & 6 & 87837 & 787 \\ 
     \hline
     2 & 7 & 78 & 5415 \\
     \hline
     3 & 545 & 778 & 7507 \\
     \hline
     4 & 545 & 18744 & 7560 \\
     \hline
     5 & 88 & 788 & 6344 \\ [1ex] 
     \hline
\end{tabular}
\end{center}

Recent advances has shown. Moreover Gusev et al. proposed high 
flexibility adaption


4. 


\end{document}



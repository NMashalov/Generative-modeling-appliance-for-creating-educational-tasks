
Методы аугментации изображений в компьютерном зрении представляют собой техники, используемые для увеличения размера и разнообразия тренировочного набора данных путем применения различных преобразований к изображениям. Целью аугментации является создание дополнительных вариаций изображений, что помогает улучшить обобщающую способность моделей машинного обучения и уменьшить риск переобучения.

Основные методы аугментации включают в себя изменение размера изображения (путем масштабирования), повороты, отражения, изменение яркости, контраста и насыщенности цветов, а также добавление шума или размытия. Дополнительно, могут применяться специфические трансформации, такие как сдвиги, обрезки или изменение геометрии изображения.

Применение методов аугментации позволяет модели машинного обучения обучаться на более разнообразных данных, что способствует повышению их устойчивости к различным условиям и изменениям в данных во время работы. Кроме того, аугментация может помочь справиться с проблемой несбалансированных классов и улучшить обобщающую способность моделей.

Эффективное использование методов аугментации требует тщательного анализа особенностей конкретной задачи и выбора соответствующих трансформаций, которые помогут улучшить качество модели без искажения смысла изображений. Кроме того, важно проводить проверку и оценку результатов аугментации с целью избежать нежелательных эффектов и обеспечить сбалансированное улучшение качества моделей компьютерного зрения.

Процедура разметки в области компьютерного зрения (computer vision) представляет собой процесс создания аннотаций или меток для изображений с целью обучения моделей машинного обучения. Этот процесс включает в себя несколько шагов и может быть выполнен как вручную, так и с использованием специализированных инструментов.

Один из основных методов разметки вручную - это создание прямоугольных или многоугольных областей (bounding boxes) вокруг объектов интереса на изображении. Эти bounding boxes обозначают положение и размер объекта на изображении. Для каждого bounding box также указывается класс объекта (например, кошка, собака, машина и т. д.).

Другой метод разметки включает создание сегментационных масок (segmentation masks), которые обозначают точные пиксели объекта на изображении. В этом случае каждый пиксель изображения помечается как принадлежащий к объекту или фону.

Для разметки изображений может использоваться специальное программное обеспечение, такое как LabelImg, VGG Image Annotator, LabelMe и другие. Эти инструменты обычно предоставляют пользователю удобный интерфейс для создания аннотаций, а также функции для экспорта аннотированных данных в форматы, подходящие для обучения моделей машинного обучения.

После завершения процесса разметки данные готовы для использования в обучении моделей компьютерного зрения, например, в обучении детекторов объектов, сегментационных моделей или других моделей, которые требуют размеченные данные для обучения.

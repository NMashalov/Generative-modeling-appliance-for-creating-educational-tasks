

\subsection{Выделение объектов}

\cite{kirillov2023segment}



Модель YOLO (You Only Look Once) представляет собой популярную архитектуру для обнаружения объектов на изображениях. Ее основной идеей является выполнение обнаружения объектов и классификации в одной сети, что делает ее быстрой и эффективной.

Порядок работы модели YOLO начинается с входного изображения, которое подается на вход нейронной сети. Затем изображение проходит через сверточные слои, которые извлекают признаки из изображения на различных уровнях абстракции.

Далее, полученные признаки пропускаются через сверточные слои, которые прогнозируют боксы (ограничивающие рамки) для объектов и их вероятности принадлежности к различным классам. Эти сверточные слои производят прогнозы на основе якорей (anchors), которые представляют разные размеры и соотношения сторон боксов.

После этого выполняется пост-обработка, включающая подавление неоднородных предсказаний (non-maximum suppression), чтобы получить финальные прогнозы объектов. Этот шаг удаляет лишние дубликаты и уверенно прогнозирует объекты с наибольшей уверенностью (confidence).

В результате работы модели YOLO получается набор боксов с классами и оценками уверенности, представляющих объекты, найденные на изображении. Эта информация может быть использована для обнаружения объектов и их классификации в реальном времени.

\subsubsection{Выделение объектов}


Алгоритм bounding box представляет собой метод в области компьютерного зрения, направленный на выделение прямоугольной области, охватывающей объекты на изображении. Основная цель алгоритма состоит в определении минимального прямоугольника, который содержит объект, сохраняя при этом минимальные потери информации.

Принцип работы алгоритма bounding box заключается в следующих шагах:

1. Подготовка изображения: Изображение предварительно обрабатывается для улучшения качества и подготовки к дальнейшему анализу.
2. Обнаружение объектов: Проводится анализ изображения с целью выявления интересующих областей, используя различные методы, такие как выделение краев, сегментация или классификация.
3. Вычисление ограничивающих рамок: Для каждого обнаруженного объекта определяется минимальный прямоугольник, который полностью охватывает его.
4. Визуализация результатов: Ограничивающие рамки визуализируются на изображении для дальнейшего анализа или использования.

Математически алгоритм bounding box может быть описан следующим образом:

Пусть \( P = \{p_1, p_2, ..., p_n\} \) — множество точек, описывающих объект на изображении.

Координаты верхнего левого угла ограничивающей рамки $\(x_{min}, y_{min}\)$ и координаты нижнего правого угла $\(x_{max}, y_{max}\)$ определяются как:

\[ x_{\text{min}} = \min_{p \in P} (p_x), \]
\[ y_{\text{min}} = \min_{p \in P} (p_y), \]
\[ x_{\text{max}} = \max_{p \in P} (p_x), \]
\[ y_{\text{max}} = \max_{p \in P} (p_y). \]

Таким образом, алгоритм bounding box позволяет эффективно выделять и описывать объекты на изображении, что является важным инструментом в области компьютерного зрения и обработки изображений.



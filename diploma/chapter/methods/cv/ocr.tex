
Оптическое распознавание символов (OCR) представляет собой процесс автоматического преобразования текста, представленного в виде изображения или сканированного документа, в электронный текстовый формат. Термин "оптическое" в данном контексте указывает на использование оптических средств, таких как камеры или сканеры, для захвата изображений символов с физических носителей, например, бумаги.

Процесс OCR включает в себя несколько этапов, начиная с захвата изображения и заканчивая распознаванием символов и созданием электронного текста. Первоначально изображение документа подвергается предварительной обработке, такой как удаление шума или коррекция искажений. Затем происходит сегментация изображения, то есть разделение его на отдельные символы или группы символов.

Далее, при помощи алгоритмов распознавания, включающих методы машинного обучения и компьютерного зрения, символы на изображении анализируются и сопоставляются с соответствующими символами из набора знаков. Этот этап включает в себя распознавание формы символов, их контекста и других характеристик, что позволяет определить, какие символы были изображены на сканированном документе.

В завершение, распознанные символы объединяются в слова, предложения и абзацы, формируя полноценный текстовый документ. Точность и эффективность процесса OCR зависят от качества изображения, используемых алгоритмов распознавания, а также от языка и структуры текста. В современных приложениях OCR широко используются в различных областях, включая сканирование документов, распознавание номеров автомобильных номеров, оптическое чтение рукописных текстов и другие приложения, где требуется автоматическое извлечение текста из изображений.

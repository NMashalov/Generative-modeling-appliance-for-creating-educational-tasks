Energy-based подходы в машинном обучении представляют собой методы моделирования, основанные на понятии энергии системы. В этом подходе модель оценивает энергию данных, а затем использует эту оценку для различных задач, таких как классификация, регрессия или генерация данных. Основная идея состоит в том, что модель стремится минимизировать энергию для реальных данных и увеличивать ее для фиктивных (сгенерированных) данных.

Пусть \( \mathbf{x} \) — наблюдаемая переменная (например, вектор признаков), а \( E(\mathbf{x}; \theta) \) — энергия, присвоенная данным \( \mathbf{x} \) моделью с параметрами \( \theta \). Тогда модель может быть описана следующим образом:

\[ E(\mathbf{x}; \theta) \]

Здесь \( \theta \) — параметры модели, которые подлежат обучению.

Energy-based модели могут быть использованы для решения различных задач. Например, в задаче классификации модель может устанавливать низкую энергию для данных из правильного класса и высокую для данных из неправильного класса. Для регрессии модель может предсказывать энергию, близкую к целевому значению.

Обучение energy-based моделей часто включает минимизацию функции потерь, которая может быть определена как разница между энергией реальных данных и сгенерированных данных. Один из часто используемых подходов - это минимизация отклонения (discrepancy) между энергиями реальных данных \( \mathbf{x} \) и сгенерированных данных \( \tilde{\mathbf{x}} \):

\[ L(\theta) = E(\mathbf{x}; \theta) - E(\tilde{\mathbf{x}}; \theta) \]

Такой подход позволяет модели стремиться к минимизации энергии для реальных данных и максимизации энергии для сгенерированных данных.

Energy-based подходы имеют широкий спектр применений и используются в различных областях, включая глубокое обучение, генеративные модели и обучение без учителя. Они представляют собой мощный инструмент для моделирования данных с использованием концепции энергии, что позволяет справляться с различными задачами в машинном обучении.

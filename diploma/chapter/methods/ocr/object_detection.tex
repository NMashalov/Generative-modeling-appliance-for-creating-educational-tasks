Поиск объектов 



\subsection{Выделение объектов}


Алгоритм bounding box представляет собой метод в области компьютерного зрения, направленный на выделение прямоугольной области, охватывающей объекты на изображении. Основная цель алгоритма состоит в определении минимального прямоугольника, который содержит объект, сохраняя при этом минимальные потери информации.

Принцип работы алгоритма bounding box заключается в следующих шагах:

1. Подготовка изображения: Изображение предварительно обрабатывается для улучшения качества и подготовки к дальнейшему анализу.
2. Обнаружение объектов: Проводится анализ изображения с целью выявления интересующих областей, используя различные методы, такие как выделение краев, сегментация или классификация.
3. Вычисление ограничивающих рамок: Для каждого обнаруженного объекта определяется минимальный прямоугольник, который полностью охватывает его.
4. Визуализация результатов: Ограничивающие рамки визуализируются на изображении для дальнейшего анализа или использования.

Математически алгоритм bounding box может быть описан следующим образом:

Пусть \( P = \{p_1, p_2, ..., p_n\} \) — множество точек, описывающих объект на изображении.

Координаты верхнего левого угла ограничивающей рамки (x_min, y_min) и координаты нижнего правого угла (x_max, y_max) определяются как:

\[ x_{\text{min}} = \min_{p \in P} (p_x), \]
\[ y_{\text{min}} = \min_{p \in P} (p_y), \]
\[ x_{\text{max}} = \max_{p \in P} (p_x), \]
\[ y_{\text{max}} = \max_{p \in P} (p_y). \]

Таким образом, алгоритм bounding box позволяет эффективно выделять и описывать объекты на изображении, что является важным инструментом в области компьютерного зрения и обработки изображений.

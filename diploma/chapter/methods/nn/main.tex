\subsection{Модель Хопфилда}

Модель Хопфилда - это нейронная сеть, предложенная Джоном Хопфилдом в 1982 году, которая является одним из первых примеров рекуррентных нейронных сетей и используется для моделирования ассоциативной памяти и ассоциативного запоминания. Эта модель вдохновлена концепцией обработки информации в мозге и базируется на принципе ассоциативного запоминания, который подразумевает возможность восстановления целевого образа (или паттерна) по части информации.

В модели Хопфилда каждый нейрон представляет собой двоичный (или биполярный) элемент, который может находиться в одном из двух состояний: активном (1) или неактивном (0). Нейроны соединены сетью симметричных связей, где каждая связь имеет ассоциативную силу между соответствующими нейронами. Эта сеть связей может быть представлена в виде симметричной матрицы весов, где элементы \(w_{ij}\) обозначают силу связи между нейронами \(i\) и \(j\).

Динамика модели Хопфилда определяется обновлением активации каждого нейрона в соответствии с функцией активации и правилом Хопфилда:

1. **Инициализация**: Начнем с заданного начального состояния нейронов.

2. **Обновление нейронов**: Нейроны обновляются параллельно и асинхронно на основе правила Хопфилда:
   \[ s_i(t+1) = \text{sign}\left(\sum_j w_{ij} s_j(t)\right) \]

3. **Повторение шага 2**: Процесс обновления нейронов повторяется до тех пор, пока система не стабилизируется в некотором состоянии или пока не произойдет сходимость к сохраненным паттернам.

Одно из ключевых свойств модели Хопфилда - это способность к ассоциативному запоминанию. После обучения модель способна восстанавливать сохраненные образы при предъявлении искаженных или неполных версий этих образов. Это происходит благодаря тому, что система восстанавливается к ближайшему сохраненному образу в пространстве состояний.

Модель Хопфилда была активно изучена как теоретически, так и экспериментально, и она оказала влияние на развитие нейроинформатики и ассоциативного запоминания. Она также служит основой для более сложных и глубоких моделей нейронных сетей, используемых в современных исследованиях машинного обучения и искусственного интеллекта.

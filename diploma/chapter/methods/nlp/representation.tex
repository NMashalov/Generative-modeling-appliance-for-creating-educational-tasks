\subsection{Лемматизация}

Лемматизация представляет собой процесс нормализации текста, целью которого является приведение слов к их базовой форме или лемме. В контексте обработки естественного языка (Natural Language Processing, NLP), лемматизация является важным этапом предварительной обработки текста, который позволяет уменьшить размер словаря и улучшить качество анализа.

Формально, лемматизация выражается в преобразовании слова \( w \) в его лемму \( \text{lemma}(w) \) с использованием правил и алгоритмов, учитывающих морфологические особенности языка. Лемма представляет собой каноническую, нормализованную форму слова, которая может быть использована для обобщения различных грамматических форм одного и того же слова.

Математически, процесс лемматизации может быть представлен как отображение слова \( w \) в его лемму \( \text{lemma}(w) \):

\[ \text{lemma}(w) = \text{лемма} \]

Например, для слова "бегу", его леммой будет слово "бежать". Лемматизация помогает уменьшить словарь слов и снизить размерность пространства признаков в текстовых данных, что положительно влияет на производительность алгоритмов обработки текста, таких как классификация или кластеризация.

Применение лемматизации часто сопровождается предварительным шагом токенизации, в котором текст разбивается на отдельные слова или токены. Это позволяет применить лемматизацию к каждому слову в тексте независимо от контекста. Лемматизация часто используется в различных областях NLP, включая информационный поиск, анализ тональности, машинный перевод и другие.

\subsection{Векторное представление}


Практически востребованной оказалась дистрибутивная гипотеза \cite{Schutze},
легшая в основу алгоритма \cite{NIPS2013_9aa42b31}.

В генеративном моделировании естественного языка, встает задача представления слов в виде векторов в многомерном пространстве, что позволяет моделировать семантические и синтаксические аспекты текста в компактной форме. Это представление, известное как "векторное вложение" или "embedding", позволяет выразить смысловые и лингвистические свойства слов, используемых в языке.

Формально, векторное вложение \( \mathbf{e}_w \) слова \( w \) представляет собой векторное представление этого слова в многомерном пространстве:

\[ \mathbf{e}_w = (e_{w1}, e_{w2}, ..., e_{wd}) \]

где \( d \) - размерность пространства вложения (число измерений), \( e_{wj} \) - \( j \)-ая компонента вектора вложения \( \mathbf{e}_w \).

Эти векторные представления обычно изучаются и извлекаются из больших корпусов текстов с использованием различных алгоритмов, таких как word2vec, GloVe (Global Vectors for Word Representation), FastText и другие. Они обладают свойством сохранения семантической близости слов в пространстве вложения: слова, которые часто встречаются в похожих контекстах, имеют близкие векторные представления.

Векторные вложения слов играют важную роль в генеративном моделировании естественного языка, так как они позволяют моделям представлять слова в виде непрерывных числовых значений, которые могут быть использованы как входные данные для алгоритмов машинного обучения. Это позволяет моделям эффективно изучать зависимости между словами и генерировать тексты семантически богатые и лингвистически осмысленные.





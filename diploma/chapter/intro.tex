\chapter{Введение}

Автоматическая постановка и дополнение обучающих задач востребованное направление в сфере образования. 
Алгоритмические методы позволяют разрешать классические проблемы образования,
включающие составление методической литературы,
пресечение недобросовестной кооперации обучающихся при  индивидуальном контроле знаний,
формирование индивидуальной образовательной траектории.


Работы также показывают успешное применение автоматической генерации для формирования индивидуальной образовательной траектории
тематически однородного, но разноуровневого по сложности. Тем не менее предложенные подходы
требуют значительных временных экспертов для создания новых методических курсов. 

Стремительное развитие генеративного моделирования в областях естественного языка \cite{radford2019language} 
и машинного зрения \cite{rombach2022highresolution}\cite{song2020generative} определили
новые подходы в известных задачам нотариального консультирования, . 


Задача работы применить методы генеративного моделирования для решения задачи. В дополнение работы 
выпускается кодовая база и обучающие данные для воспроизведения эксперимента и проведения позволяет На практике 


В работах ... уже

Появляются возможности генерации задач тематических задач по интересам обучающегося, обращихся к предметному опыту и интуиции.

объективного оцениванияне на  




Работа состоит из 5 частей. 

В первой части работы будут опсианы 
\chapter{Проделанная работа}
Работа по созданию ассистента выполнялась поэтапно. 

Исходным этапом работы являлось создание корпуса педагогических задач, извлеченных из открытых источников российских учебников. Процесс сбора данных осуществлялся при помощи технологий оптического распознавания символов (OCR), включая методы, разработанные в рамках данного исследования.
 В дальнейшем полученный корпус

Глава разделена на части по 

В главе приведено описание поставленных по трем направлениям. Генерация текста задачи, сопровождающей иллюстрации и  

\section{Формирование Иллюстрации}

Данные были собраны из открытых источников \cite{libmipt}\cite{mathedu}. 


В состав датасета входит более десятка тысяч аннотированных изображений  
Результат моделирования предоставлены 
на открытых ресурсах\footnote{
\url{https://github.com/NMashalov/Generative-modeling-appliance-for-creating-educational-tasks}
и \url{https://huggingface.co/datasets/NMashalov/task_illustrations_dataset}
}

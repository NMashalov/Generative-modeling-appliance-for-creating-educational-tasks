\section{Виды задач}

Научный метод требует воспроизводимости результатов. Несмотря на многообразность педагогической деятельности, 
она также содержит закономерности, учтение которых повышает качество образовательного процесса. Многие из них исходят из биологической и социальной природы человек: потребность во сне, еде и отдыхе, распределение социума по интересам, любопытство и стремление к новизне. Некоторые из закономерностей создаются искусственно и поддерживаются для достижения воспитательного эффекта, таковыми например повседневными спортивные и творческие кружки, забота о чистоте национального языка, патриотическое воспитание. Есть и закономерности, которые существуют только "на местах". Это могут быть погодные или социальные условия, которые могут быть эффективно использованы интересов учащихся.

Закономерности позволяют стабилизировать образование в целом, гарантировать пополнение кадровой базы по социально востребованным специальностям, планировать долгосрочный рост государства.

Закономерности позволяют находить оптимальные в целом. Большим трендом является перспектива индивидуального обучения специализирующая на 


\section{Об особенностях массового образования}

Проффессия подразумевает решение сложных задач. Задач, которые не имеют общеизвестного алгоритмически, а
Именно поиск и проведение педагогического эксперимента выращивает сильные преподавателей.


Массовое образование является . Задачи образования многочисленны:
- управленческая. Образование воспитывает
- источником рабочих мест 
-

Таким образом, индивидуальное 

Основными проблемами является

\subsection{О примате речи}

Возможностью задать вопрос. Книга вынуждает искать симметрии 



\subsection{Об устаревание методической литературы}

Появляются новые формы

\subsection{О росте сложности заданий}

- алгоритмическая сложность 

Вопрос в заданиях на чтениях побуждают к поиску закономерностей.


Возможность опоры на вспомогательный пример. Ассоциативная память не способна

\subsection{О индивидуальных случаях}

Массовое образование подразумевает эффективную адаптацию подход на месте.

Учитель .
Это выбор

\section{О дисциплине}

Нарушение дисциплины является утешением для многих учеников. Многие ученики нарушают 

Дисциплина задается ее упорядоченность, действия выходящие за дисциплины должны быть окутаны 

Дисциплина также может инструментом нарушения


\section{О прекрасном}

Автор видит основную проблему образования в ее неественной подаче.

Проблема прослежвиается на всех этапах образования
- 
- студенты пишут 

Школа 

Я опишу свое собственное видение

Становились предметами осязания. Учиться объяснять явления с помощью приобретенных знаний

Как правила хорошие задачи происходят из ситуации, наполненных эмоциональными переживаниями

Автор работы поделится случаем из собственной жизни для иллюстрации метода:
Сосулька упала на щеку дедушки и раздробило челюстную ксоть

Р


Задача 

В первую очередь задача



\section{Содержание педагогической задачи}



Преимущественно задачи школьного образования покрывают потребности управления. Школьный курс учат выполнять приказы, которые по тем или иным причинам не получилось систематизировать в программный код. Потому основным наполнением задачи является алгоритмическая составляющая с небольшим изменением под потребность. 

Подход имеет преимущества задачи закрывают некомпетенции образовательного коллектива, служат инструментом власти, 

Таким образом, наилучшим

Педагогическая задача инструмент воспитания. Задачи. 


Математический метод качественно отличается от школьного, но требует больших компетенций преподавателей.
Причиной потребности является необходимость в развитие способностей для 





\subsection{О жизненных ситуациях непокрытых}


\subsection{Контекстизация задачи}
Известная работа \сite{miller1956magical} 

С развития 

Основные формы представления

Алгоритмическая база задач


Социальная сторона задачи. 